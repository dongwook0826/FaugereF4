\chapter{Faug\`ere's F4 Algorithm}

\begin{definition}[Monomial set]
    \label{def:MonomialSet}
    \lean{monomial_set}
    \leanok
    The \textbf{monomial set} of a polynomial \(f\) is the set of monomials with
    nonzero coefficient in \(f\), and is denoted by \(\monset(f)\). For \(K
    \subseteq k[x_1, \cdots, x_n]\), we define as
    \[\monset(K) = \bigcup_{f \in K} \monset(f).\]
\end{definition}

%\begin{latexonly}
\begin{definition}[Symbolic preprocessing]
    \label{def:SymbolicPreprocessing}
    \uses{def:LeadingMonomialCoeff, def:MonomialSet}
    \lean{SymbProcStruct, symbolic_preprocess_step, symbolic_preprocess}
    \leanok
    ~\\
    
    Input: \(L\), \(G = \sset{f_1, \cdots, f_t}\)
    (two finite sets of polynomials)

    Output: \(H\) (a finite set of polynomial containing \(L\))

    \begin{itemize}
        \item[] \(H := L\)
        \item[] \(done := \leadmon(H)\)
        \item[] while \(done \ne \monset(H)\):
            \begin{itemize}
                \item[] \(x^\beta := \max_{<} (\monset(H) \setm done)\)
                \item[] \(done := done \cup \sset{x^\beta}\)
                \item[] if \(\exists g \in G\) such that \(\leadmon(g) \mid x^\beta\):
                    \begin{itemize}
                        \item[] \(g :=\) one such choice of \(g\)
                        \item[] \(H := H \cup \set{\frac{x^\beta}{\leadmon(g)} g}\)
                    \end{itemize}
            \end{itemize}
        return \(H\)
    \end{itemize}

    The algorithm above with input \(L\) and \(G\) terminates and obtains as an
    output a set of polynomials \(H\) satisfying the following two properties:
    \begin{enumerate}[(i)]
        \item \(L \subseteq H\), and
        \item whenever \(x^\beta\) is a monomial in some \(f \in H\), and for
        some \(g \in G\) its leading monomial \(\leadmon(g)\) divides
        \(x^\beta\), then \(\frac{x^\beta}{\leadmon(g)} g \in H\).
    \end{enumerate}
\end{definition}
%\end{latexonly}

%\begin{htmlonly}
%    \paragraph{Symbolic preprocessing}
%
%    Input: $L$, $G = \sset{f_1, \cdots, f_t}$
%    (two finite sets of polynomials)
%
%    Output: $H$ (a finite set of polynomial containing $L$)
%
%    \begin{itemize}
%        \item $H := L$
%        \item $done := \leadmon(H)$
%        \item while $done \ne \monset(H)$:
%        \begin{itemize}
%            \item $x^\beta := \max_{<} (\monset(H) \setm done)$
%            \item $done := done \cup \sset{x^\beta}$
%            \item if $\exists g \in G$ such that $\leadmon(g) \mid x^\beta$:
%                \begin{itemize}
%                    \item $g :=$ one such choice of $g$
%                    \item $H := H \cup \set{\frac{x^\beta}{\leadmon(g)} g}$
%                \end{itemize}
%        \end{itemize}
%        return $H$
%    \end{itemize}
%\end{htmlonly}

\iffalse % moved into def:SymbolicPreprocessing
\begin{theorem}[Result of symbolic preprocessing]
    \label{thm:SymbolicPreprocessingResult}
    \uses{def:SymbolicPreprocessing}
\end{theorem}
\fi

\begin{definition}[Faug\`ere's F4 algorithm]
    \label{def:FaugereF4}
    \uses{def:LeadingMonomialCoeff, def:SymbolicPreprocessing}
    \lean{F4}
    \leanok
    ~\\
    
    Input: \(F = \sset{f_1, \cdots, f_s}\)
    (a generating set of polynomials of an ideal)

    Output: \(G\) (a Gr\"obner basis of the ideal, containing \(F\))

    \begin{itemize}
        \item[] \(G := F\)
        \item[] \(t := s\)
        \item[] \(B := \{\{i, j\} \mid 1 \le i < j \le s\}\)
        \item[] while \(B \ne \emptyset\):
            \begin{itemize}
                \item[] \(B' := \textrm{a nonempty subset of } B\)
                \item[] \(B := B \setm B'\)
                \item[] \(L := \left\{\frac{\lcm(\leadmon(f_i), \leadmon(f_j))}{\leadterm(f_i)} f_i \mid \{i, j\} \in B' \right\}\)
                \item[] \(H := SymbolicPreprocessing(L, G)\)
                \item[] \(M := (\mathrm{coeff}(h_k, x^{\alpha_\ell}))_{k, \ell}\)\\
                (the matrix of coefficients of H; \(x^{\alpha_\ell}\) sorted under monomial order)
                \item[] \(N := \textrm{row reduced echelon form of } M\)
                \item[] \(N^+ := \{n \in \mathrm{rows}(N) \mid \leadmon(n) \notin \gen{\leadmon(\mathrm{rows}(N))}\}\)
                \item[] for \(n \in N^+\):
                    \begin{itemize}
                        \item[] \(t := t + 1\)
                        \item[] \(f_t := \textrm{the polynomial form of } n\)
                        \item[] \(G := G \cup \{f_t\}\)
                        \item[] \(B := B \cup \{\{i, t\} \mid 1 \le i < t\}\)
                    \end{itemize}
            \end{itemize}
        return \(G\)
    \end{itemize}
\end{definition}

\begin{theorem}[Result of F4]
    \label{def:FaugereF4Result}
    \uses{def:FaugereF4, thm:RefinedBuchbergerCriterion}
    \lean{F4_groebner}
    \leanok
    The output \(G'\) of Faug\`ere's F4 algorithm is a Gr\"obner basis of the ideal
    generated by the input polynomial set \(G\). In particular, the output
    satisfies the refined Buchberger criterion; i.e. every S-polynomial within
    \(G'\) reduces to zero over \(G'\).
\end{theorem}