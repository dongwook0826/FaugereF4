\chapter{Faug{\`e}re F4 Algorithm}

\begin{definition}[Symbolic preprocessing]
    Input: \(L\), \(G = \sset{f_1, \cdots, f_t}\)
    (two finite sets of polynomials) \\
    Output: \(H\) (a finite set of polynomial containing \(L\)) \\

    \begin{itemize}
        \item \(H := L\)
        \item \(done := \leadmon(H)\)
        \item while \(done \ne \monset(H)\):
        \begin{itemize}
            \item \(x^\beta := \max_{<} (\monset(H) \setm done)\)
            \item \(\done := \done \cup \sset{x^\beta}\)
            \item if \(\exists g \in G\) such that \(\leadmon(g) | x^\beta\):
                \begin{itemize}
                    \item \(g :=\) one such choice of \(g\)
                    \item \(H := H \cup \set{\frac{x^\beta}{\leadmon(g)} g}\)
                \end{itemize}
        \end{itemize}
        return \(H\)
    \end{itemize}
\end{definition}

\begin{theorem}[Result of symbolic preprocessing]
    The algorithm above with input \(L\) and \(G\) terminates and obtains as an
    output a set of polynomials \(H\) satisfying the following two properties:
    \begin{enumerate}[label=\textbf{(\roman*)}]
        \item \(L \subseteq H\), and
        \item whenever \(x^\beta\) is a monomial in some \(f \in H\), and for
        some \(g \in G\) its leading monomial \(\leadmon(g)\) divides
        \(x^\beta\), then \(\frac{x^\beta}{\leadmon(g)} g \in H\).
    \end{enumerate}
\end{theorem}

%\begin{definition}[Faug{\`e}re F4 algorithm framework]

%\end{definition}
