\chapter{Buchberger's Criterion}

\begin{definition}[Multivariate division algorithm]
    \label{def:MvDivisionAlgorithm}
    \uses{def:LeadingMonomialTerm}
    ~\\

    Input: divisor set \(\{f_1, \cdots, f_s\}\) and dividend \(f\)

    Output: quotients \(q_1, \cdots, q_s\) and remainder \(r\)

    \begin{itemize}
        \item \(\forall\,i,\,q_i := 0;\, r := 0\)
        \item \(p := f\)
        \item while \(p \ne 0\):
            \begin{itemize}
                \item \(i := 1\)
                \item \(DivisionOccured := False\)
                \item while \(i \le s\) and not \(DivisionOccured\):
                    \begin{itemize}
                        \item if \(\leadterm(f_i) | \leadterm(p)\):
                            \begin{itemize}
                                \item \(q_i := q_i + \leadterm(p) / \leadterm(f_i)\)
                                \item \(p := p - (\leadterm(p) / \leadterm(f_i))f_i\)
                            \end{itemize}
                        \item else:
                            \begin{itemize}
                                \item \(i := i + 1\)
                            \end{itemize}
                    \end{itemize}
                \item if not \(DivisionOccured\):
                    \begin{itemize}
                        \item \(r := r + \leadterm(p)\)
                        \item \(p := p - \leadterm(p)\)
                    \end{itemize}
            \end{itemize}
        return \(q_1, \cdots, q_s, r\)
    \end{itemize}

\end{definition}

\begin{definition}[S-polynomial]
    \label{def:SPolynomial}
    \uses{def:LeadingMonomialTerm}
    Define the \textbf{least common multiple} \(\gamma = \lcm(\alpha, \beta)\)
    of two monomials \(\alpha, \beta\) as \(\gamma_i = \max(\alpha_i, \beta_i)\).
    The \textbf{S-polynomial} of two polynomials \(f\) and \(g\), given a monomial order,
    is
    \[S(f, g) = \frac{\lcm(\leadmon(f), \leadmon(g))}{\leadterm(f)} f
              - \frac{\lcm(\leadmon(f), \leadmon(g))}{\leadterm(g)} g.\]
    We say each part of above, i.e.
    \[\frac{\lcm(\leadmon(f), \leadmon(g))}{\leadterm(f)} f \quad \textrm{and} \quad
      \frac{\lcm(\leadmon(f), \leadmon(g))}{\leadterm(g)} g,\]
    the \textbf{S-pair} of \(f\) and \(g\).
\end{definition}

\begin{theorem}[Buchberger's criterion]
    \label{thm:BuchbergerCriterion}
    \uses{def:GroebnerBasis, def:MvDivisionAlgorithm, def:SPolynomial}
    Let \(I \idealle k[x_1, \cdots, x_n]\). Then a basis \(G = \{g_1, \cdots, g_t\}\)
    is a Gr\"obner basis of \(I\) iff the remainder of each \(S(g_i, g_j)\)(\(i \ne j\))
    in long division by \(G\) is zero.
\end{theorem}

\begin{definition}[Standard representation]
    \label{def:StandardRepresentation}
    \uses{def:LeadingMonomialTerm}
    For \(G = \{g_1, \cdots, g_t\} \subseteq k[x_1, \cdots, x_n]\) and \(f \in k[x_1, \cdots, x_n]\),
    a \textbf{standard representation} of \(f\) by \(G\) is, if exists, an equality
    \[f = \sum_{k=1}^t A_k g_k, \quad A_k \in k[x_1, \cdots, x_n],\]
    where \(A_k g_k = 0\) or \(\leadmon(f) \ge \leadmon(A_k g_k)\) for every \(1 \le k \le t\).
    If such a standard representation exists, we say \(f\) \textbf{reduces to zero modulo} \(G\)
    and notate it as
    \[f \to_G 0.\]
\end{definition}

\begin{theorem}[Refinement of Buchberger's criterion]
    \label{thm:RefinedBuchbergerCriterion}
    \uses{def:GroebnerBasis, thm:BuchbergerCriterion, def:StandardRepresentation}
    Let \(I \idealle k[x_1, \cdots, x_n]\). Then a basis \(G = \{g_1, \cdots, g_t\}\)
    is a Gr\"obner basis of \(I\) iff \(S(g_i, g_j) \to_G 0\) for each pair of \(i \ne j\).
\end{theorem}