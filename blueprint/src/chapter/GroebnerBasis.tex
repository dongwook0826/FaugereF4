\chapter{Gr\"obner Basis}

Let \(f \in k[x_1, \cdots, x_n] \setm \sset{0}\).

\begin{definition}[Leading monomial \& leading term] % [Cox, Def 2.2.7.]
    \label{def:LeadingMonomialTerm}
    The \textbf{leading monomial} \(\leadmon(f)\) of \(f\) is the monomial in
    \(f\) maximum under the fixed monomial order. The \textbf{leading term}
    \(\leadterm(f)\) of \(f\) is the multiple of \(\leadmon(f)\) by its
    coefficient in \(f\).
\end{definition}

\begin{definition}[Monomial ideal] % [Cox, Def 2.4.1.]
    \label{def:MonomialIdeal}
    An ideal \(I \idealle k[x_1, \cdots, x_n]\) is a \textbf{monomial ideal}
    if for some subset of exponents \(A \subseteq \Z_{\ge 0}^n\),
    \[I = \sgen{x^\alpha : \alpha \in A}.\]
\end{definition}

\begin{theorem}[Dickson's lemma] % [Cox, Thm 2.4.5.]
    \label{thm:DicksonLemma}
    \uses{def:MonomialIdeal}
    For any monomial ideal \(I = \sgen{x^\alpha : \alpha \in A} \idealle
    k[x_1, \cdots, x_n]\), there exists a finite subset \(A' \subseteq A\)
    such that \(I = \sgen{x^\alpha : \alpha \in A'}\).
\end{theorem}

\begin{definition}[Ideal of leading terms] % [Cox, Def 2.5.1.]
    \label{def:LeadTermIdeal}
    \uses{def:LeadingMonomialTerm, def:MonomialIdeal}
    Let \(I \idealle k[x_1, \cdots, x_n]\) a nontrivial ideal. The
    \textbf{ideal of leading terms} of \(I\) is the ideal generated by leading
    terms of each \(f \in I \setm \sset{0}\). Namely,
    \[\sgen{\leadterm(I)}
    = \sgen{\leadterm(f) : f \in I \setm \sset{0}}.\]
    This is equivalent to being generated by leading monomials, i.e. the above
    ideals are equal to
    \[\sgen{\leadmon(I)}
    = \sgen{\leadmon(f) : f \in I \setm \sset{0}}.\]
\end{definition}

\begin{definition}[Gr\"obner basis] % [Cox, Def 2.5.5.]
    \label{def:GroebnerBasis}
    \uses{def:LeadTermIdeal}
    A finite subset \(G = \sset{g_1, \cdots, g_t} \ne \sset{0}\) of \(I \idealle
    k[x_1, \cdots, x_n]\) is said to be a \textbf{Gr{\"o}bner basis} of \(I\) if
    \[\sgen{\leadmon(I)} = \sgen{\leadmon(G)}
    = \sgen{\leadmon(g_1), \cdots, \leadmon(g_t)}.\]
\end{definition}

\begin{definition}[Monomial set]
    \label{def:MonomialSet}
    The \textbf{monomial set} of a polynomial \(f\) is the set of monomials with
    nonzero coefficient in \(f\), and is denoted by \(\monset(f)\). For \(K
    \subseteq k[x_1, \cdots, x_n]\), we define as
    \[\monset(K) = \bigcup_{f \in K} \monset(f).\]
\end{definition}