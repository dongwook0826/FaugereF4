\chapter{Gr\"obner Basis}

Here we fix a field \(k\) of coefficients, and a monomial order \(\le\) on
\(k[x_1, \cdots, x_n]\). Let \(f \in k[x_1, \cdots, x_n] \setm \sset{0}\).

\begin{definition}[Monomial order]
    \label{def:MonomialOrder}
    \lean{MonomialOrder} % already in mathlib
    \leanok
    A \textbf{monomial order} on \(k[x_1, \cdots, x_n]\) is a total order \(<\)
    on \(\Z_{\ge 0}^n\) satisfying:
    \begin{enumerate}[(i)]
        \item if \(\alpha < \beta\), then \(\alpha + \gamma < \beta + \gamma\)
        for any \(\gamma \in \Z_{\ge 0}^n\);
        \item \(<\) is a well-ordering.
    \end{enumerate}
\end{definition}

\begin{definition}[Leading monomial, leading coefficient, and leading term] % [Cox, Def 2.2.7.]
    \label{def:LeadingMonomialCoeff}
    \uses{def:MonomialOrder}
    \lean{leading_monomial, leading_monomial', leading_coeff'}
    \leanok
    The \textbf{leading monomial} \(\leadmon(f)\) of \(f\) is the monomial in
    \(f\) maximum under the fixed monomial order. The \textbf{leading coefficient}
    \(\leadcoeff(f)\) of \(f\) is the coefficient of \(\leadmon(f)\) in \(f\).
    The \textbf{leading term} of \(f\) is then simply the \(\leadcoeff(f)\)-multiple
    of \(\leadmon(f)\).
\end{definition}

\begin{definition}[Monomial ideal] % [Cox, Def 2.4.1.]
    \label{def:MonomialIdeal}
    \lean{monomial_ideal}
    \leanok
    An ideal \(I \idealle k[x_1, \cdots, x_n]\) is a \textbf{monomial ideal}
    if there exists a subset of exponents \(A \subseteq \Z_{\ge 0}^n\) such that
    \[I = \sgen{x^\alpha : \alpha \in A}.\]
\end{definition}

\iffalse % This project doesn't use Dickson's lemma.
\begin{theorem}[Dickson's lemma] % [Cox, Thm 2.4.5.]
    \label{thm:DicksonLemma}
    \uses{def:MonomialIdeal}
    For any monomial ideal \(I = \sgen{x^\alpha : \alpha \in A} \idealle
    k[x_1, \cdots, x_n]\), there exists a finite subset \(A' \subseteq A\)
    such that \(I = \sgen{x^\alpha : \alpha \in A'}\).
\end{theorem}
\fi

\iffalse
% Leading-term ideal is equivalent to monomial ideal generated by leading monomials;
% this definition is redundant.
\begin{definition}[Ideal of leading terms] % [Cox, Def 2.5.1.]
    \label{def:LeadTermIdeal}
    \uses{def:LeadingMonomialCoeff, def:MonomialIdeal}
    Let \(I \idealle k[x_1, \cdots, x_n]\) a nontrivial ideal. The
    \textbf{ideal of leading terms} of \(I\) is the ideal generated by leading
    terms of each \(f \in I \setm \sset{0}\). Namely,
    \[\sgen{\leadterm(I)}
    = \sgen{\leadterm(f) : f \in I \setm \sset{0}}.\]
    This is equivalent to being generated by leading monomials, i.e. the above
    ideals are equal to
    \[\sgen{\leadmon(I)}
    = \sgen{\leadmon(f) : f \in I \setm \sset{0}}.\]
\end{definition}
\fi

\begin{definition}[Gr\"obner basis] % [Cox, Def 2.5.5.]
    \label{def:GroebnerBasis}
    \uses{def:LeadingMonomialCoeff, def:MonomialIdeal}
    \lean{is_groebner}
    \leanok
    A finite subset \(G = \sset{g_1, \cdots, g_t} \ne \sset{0}\) of \(I \idealle
    k[x_1, \cdots, x_n]\) is said to be a \textbf{Gr{\"o}bner basis} of \(I\) if
    \[\sgen{\leadmon(I)} = \sgen{\leadmon(G)}
    = \sgen{\leadmon(g_1), \cdots, \leadmon(g_t)}.\]
\end{definition}